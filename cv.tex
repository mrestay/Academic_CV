%------------------------------------
% Dario Taraborelli
% Typesetting your academic CV in LaTeX
%
% URL: http://nitens.org/taraborelli/cvtex
% DISCLAIMER: This template is provided for free and without any guarantee 
% that it will correctly compile on your system if you have a non-standard  
% configuration.
% Some rights reserved: http://creativecommons.org/licenses/by-sa/3.0/
%------------------------------------

%!TEX TS-program = xelatex
%!TEX encoding = UTF-8 Unicode

\documentclass[11pt, a4paper]{article}
\usepackage{fontspec} 

% DOCUMENT LAYOUT
\usepackage{geometry} 
\geometry{a4paper, textwidth=5.5in, textheight=8.5in, marginparsep=7pt, marginparwidth=.6in}
\setlength\parindent{0in}

% FONTS
\usepackage[usenames,dvipsnames]{xcolor}
\usepackage{xunicode}
\usepackage{xltxtra}
\defaultfontfeatures{Mapping=tex-text}
%\setromanfont [Ligatures={Common}, Numbers={OldStyle}, Variant=01]{Linux Libertine O}
%\setmonofont[Scale=0.8]{Monaco}
%%% modified by Karol Kozioł for ShareLaTeX use
\setmainfont[
  Ligatures={Common}, Numbers={OldStyle}, Variant=01,
  BoldFont=LinLibertine_RB.otf,
  ItalicFont=LinLibertine_RI.otf,
  BoldItalicFont=LinLibertine_RBI.otf
]{LinLibertine_R.otf}
\setmonofont[Scale=0.8]{DejaVuSansMono.ttf}

% ---- CUSTOM COMMANDS
\chardef\&="E050
\newcommand{\html}[1]{\href{#1}{\scriptsize\textsc{[html]}}}
\newcommand{\pdf}[1]{\href{#1}{\scriptsize\textsc{[pdf]}}}
\newcommand{\doi}[1]{\href{#1}{\scriptsize\textsc{[doi]}}}
% ---- MARGIN YEARS
\usepackage{marginnote}
\newcommand{\amper{}}{\chardef\amper="E0BD }
\newcommand{\years}[1]{\marginnote{\scriptsize #1}}
\renewcommand*{\raggedleftmarginnote}{}
\setlength{\marginparsep}{7pt}
\reversemarginpar

% HEADINGS
\usepackage{sectsty} 
\usepackage[normalem]{ulem} 
\sectionfont{\mdseries\upshape\Large}
\subsectionfont{\mdseries\scshape\normalsize} 
\subsubsectionfont{\mdseries\upshape\large} 

% PDF SETUP
% ---- FILL IN HERE THE DOC TITLE AND AUTHOR
\usepackage[%dvipdfm, 
bookmarks, colorlinks, breaklinks, 
% ---- FILL IN HERE THE TITLE AND AUTHOR
	pdftitle={Mateo Restrepo - Cv},
	pdfauthor={Mateo Restrepo},
	pdfproducer={http://nitens.org/taraborelli/cvtex}
]{hyperref}  
\hypersetup{linkcolor=blue,citecolor=blue,filecolor=black,urlcolor=MidnightBlue} 

% DOCUMENT
\begin{document}
{\Huge Mateo Restrepo Ayala}\\[1cm]
Physics Department Universidad de los Andes\\
Carrera 1 \#18A-12\\
%Personal Address: Avenida Carrera 72 \#24B-34\\
Bogotá - Colombia\\[.2cm]
Mobile: \texttt{+57 314-257-7884}\\[.2cm]
Email: \href{mailto:m.restrepo11@uniandes.edu.co}{m.restrepo11@uniandes.edu.co}\\ 
Github: \url{https://github.com/m-restrepo11}\\
Nationality:  Colombian\\


%%\hrule
\section*{\textcolor{BrickRed}{Education}}
\noindent
\years{2017-2019}\textbf{\textsc{M.Sc.} in Physics}, Universidad de los Andes, Bogotá, Colombia.\\
{GPA:} 4.45/5\\
\years{2012-2016}\textbf{\textsc{B.Sc.} in Physics}, Universidad de los Andes, Bogotá, Colombia.\\ 
GPA: 4.38/5. \textsc{Thesis:} \textit{"Investigation of the Physics of the Sorkin-Johnston state as a distinguished vacuum state for quantum field theory in curved Spacetimes"}\\
\years{2013-2016}\textbf{\textsc{Minor} in Mathematics}, Universidad de los Andes, Bogotá, Colombia.\\
{GPA:} 4.40/5

%%\hrule
\section*{\textcolor{BrickRed}{Current Position}}
\noindent
\years{2017-2019}\textbf{\textsc{Graduate research assistant}}, Theoretical physics group at physics department Universidad de los Andes, Bogotá, Colombia. 

%%\hrule
\section*{\textcolor{BrickRed}{Areas of Interest}}
  Quantum field theory • Condensed matter \& many-body physics • Machine learning  • Information theory • Complex systems
  
\section*{\textcolor{BrickRed}{Publications}}
\years{2017} A random number generator based on polarized photons. \textit{Emergent Scientist \textbf{Vol 1}- 2.}
%%\hrule

\section*{\textcolor{BrickRed}{Programming Skills}}
\textbf{Systems:} Linux, MsWindows.\\  \textbf{Development:} Python, C, C++, Java, Bash.\\
\textbf{Software:} Mathematica, \LaTeX.\\ \textbf{Tools:} Tensorflow, Pandas, Sckit-learn.
%%\hrule
\newpage
\section*{\textcolor{BrickRed}{Research Experience}}
%\years{2017} \textsc{Graudate research assistant.}, Physics department - Universidad de los Andes.\\
%\years{2017} \textsc{Bohmnian mechanics:  A Hydrodynamic Approach to Quantum Mechanics and Bose-Einstein condensate dynamics.} Final work for the advanced quantum mechanics course.\\
\years{2018} \textbf{\textsc{Summer research internship}}. Under the supervision of Andrés F. Reyes-Lega. \textit{Quanutm Quenches and two-dimensional electrodynamics: Dirac operator spectrum evolution and chiral anomaly.} Universidad de los Andes.\\
\years{2017} \textbf{\textsc{Summer research internship}}. Under the supervision of Andrés F. Reyes-Lega. \textit{Computation of the entanglement entropy of a causal diamond in 1+1 dimensions with the Sorkin-Johnston state.}, Universidad de los Andes.\\
\years{2017}\textbf{\textsc{Research project}} \emph{Quantum state tomography of entangled qubit pairs}, Quantum optics group, Universidad de los Andes.\\
\years{2018} \textbf{\textsc{Research grant}} \emph{Schwinger model: spectrum, dynamics and quantum quenches}, Seed project grant, science faculty, Universidad de los Andes.



%%\hrule
\section*{\textcolor{BrickRed}{Languages}}
Spanish: Native \\ 
English: Fluent, \textsc{TOEFL iBT:} 108/120 \\
German: Basic

 
\section*{\textcolor{BrickRed}{Teaching \& professional Experience}}

\years{2017-2019} \textbf{\textsc{Graduate teaching assistant:}} Experimental Physics I, Experimental Physics II ,  Physics I - Recitation Section, Physics II - Recitation Section.\\
\years{2017} \textbf{\textsc{Organizer and lecturer:}} Graduate Student Seminar in quantum field theory .\\
\years{2013-2015} \textbf{\textsc{Teaching assistant:}} Clinic for Problem-Solving: an undergraduate problem solving lab. Physics Department Universidad de los Andes.\\
\years{2013-2016}\textbf{\textsc{Undergraduate teaching assistant:}} Honors Differential Calculus, Honors Vector Calculus, Physics I, Physics II, Classical Mechanics, Quantum Mechanics I, Quantum Mechanics II

%\hrule
\section*{\textcolor{BrickRed}{Grants, honors \& awards}}
\noindent
\years{2018} \textbf{\textsc{Seed project grant}} financing research projects and attendance to academic events. Science faculty, Universidad de los Andes.\\
\years{2017} \textbf{\textsc{Teaching assistantship}} covering tuition fees for M.Sc. studies. Physics department, Universidad de los Andes.\\
\years{2012} \textbf{\textsc{Alberto Magno grant}} for undergraduate tuition fee. Universidad de los Andes.\\
\years{2012} Best score in the Colombian national exam for High-school ICFES.


%%\hrule
\section*{\textcolor{BrickRed}{Conferences, Schools \& Workshops}}
\noindent

\years{2018} \emph{Q-Turn: changing paradigms in quantum science}. UFSC, Florianopolis, Brazil.\\
\years{2018} \emph{Ninth School of Mathematical Physics:Topological order and beyond}. Universidad de los Andes.\\
\years{2017} \emph{Workshop on Dynamic of quantum systems out of equilibrium.} Universidad de los Andes.\\
\years{2017} \emph{Minicourse on machine learning for many-body physics}. ICTP-SAIFR, Sao Paulo, Brazil.\\
\years{2017} \emph{Tenth Summer School on Geometric, Algebraic and Topological Methods for Quantum Field Theory}. Universidad de los Andes, Villa de Leyva, Colombia.\\
\years{2016} \emph{Journeys into Theoretical Physics}. IFT-Perimeter-SAIFR, Sao Paulo, Brazil.\\
\years{2016} \emph{Eight School of Mathematical Physics: Random Geometries}. Universidad de los Andes.\\
\years{2016} \emph{Member of the first Colombian team at the International Physicist's Tournament}. Société Française de Physique, Paris, France.\\
\years{2015} \emph{Andean School on Gravity and Cosmology}. Universidad de los Andes.\\
\years{2015} \emph{Theoretical Physics School on Quantum Field Theory}. Universidad del Norte y Universidad del Atlantico, Barranquilla, Colombia.\\
\years{2015} \emph{Ninth Summer School on Geometric, Algebraic and Topological Methods for Quantum Field Theory}. Universidad de los Andes, Villa de Leyva, Colombia.\\
\years{2015} \emph{Workshop on Mathematical Structure and Foundations of Quantum Physics}. Universidad de los Andes.\\
\years{2015} \emph{International Year of Light}. Universidad Nacional de Colombia \& Universidad de los Andes.\\
\years{2015} \emph{Seventh School of Mathematical Physics: Topological Quantum Matter From Theory to Applications}. Universidad de los Andes.


%%\hrule
\section*{\textcolor{BrickRed}{Posters, Talks \& Seminars}}

\years{2018} \textbf{\textsc{Poster}} \emph{Lattice Schwinger model: Spectrum, dynamics and quantum quenches} Q-Turn quantum information workshop, UFSC, Florianopolis, Brazil.\\
\years{2108} \textbf{\textsc{Seminar}} \emph{Schwinger model continuum and discrete formulations} Quantum field theory and mathematical physics seminar at physics department, Universidad de los Andes.\\
\years{2108} \textbf{\textsc{Seminar}} \emph{Quantum quench dynamics in the Ising model and dynamical phase transitions} Quantum field theory and mathematical physics seminar at physics department, Universidad de los Andes.\\
\years{2017} \textbf{\textsc{Seminar}} \emph{2 dimensional conformal invariance.} Quantum field theory and mathematical physics seminar at physics department, Universidad de los Andes.\\\textbf{\textsc{Seminar}} \emph{The cosmological constant problem.} Theoretical Physics Seminar at physics department, Universidad de los Andes.\\
\years{2016}\textbf{\textsc{Talk}} \emph{Beyond the dark Universe? $f(R)$ gravitational alternatives.} Astroparticle physics course talks, Universidad de los Andes.\\
\years{2016} \textbf{\textsc{Seminar}} \emph{Field theory in curved spacetimes.}Theoretical Physics Seminar at physics department, Universidad de los Andes.\\
\years{2016} \textbf{\textsc{Seminar}} \emph{The Sorkin-Johnston state and entanglement entropy of bounded spacetime regions.} Quantum Field Theory Seminar at physics department, Universidad de los Andes.\\
\years{2016} \textbf{\textsc{Seminar}} \emph{Aspects of Field Quantization and the Sorkin-Johnston state for the Klein-Gordon field.} Physics Seminar by Students at physics department, Universidad de los Andes.\\
\years{2015} \textbf{\textsc{Seminar}}\emph{Stellar Equilibrium in White Dwarf Stars and the Chandrasekhar mass}, Theoretical Physics Seminar at physics department, Universidad de los Andes.\\
\years{2015} \textbf{\textsc{Seminar}} \emph{General Relativity and Gravitational Waves}, Physics Seminar by Students at physics department, Universidad de los Andes.


\noindent



%%\hrule
%\section*{Graduate Coursework}
%\years{2015} \textsc{Spacetime structure.}\\
%\years{2015} \textsc{Information theory for physicists.}\\
%\years{2015} \textsc{General relativity and cosmology.}\\
%\years{2016} \textsc{Quantum filed theory.}\\
%\years{2016} \textsc{Asymptotic analysis.}\\
%\years{2016} \textsc{Astroparticle physics and cosmology.}\\
%\years{2017} \textsc{Advanced quantum mechanics.}\\
%\years{2017} \textsc{Advanced analytical mechanics.}\\
%\years{2017} \textsc{Advanced laboratory.}\\
%\years{2017} \textsc{Group theory in quantum mechanics}

%%\hrule
%\section*{Undergraduate Work}
%\years{2014} \textsc{Experimental Verification for the Three Kirchhoff Laws of Electromagnetic Radiation}, Final work for the Intermediate Laboratory course.\\
%\years{2015} \textsc{Gravitational Radiation: Theoretical Foundations and Detection Methods}, Final work for the Spacetime structure course.\\
%\years{2015} \textsc{F(R) theories as modifications to General Relativity}, Final work for the general relativity course.\\
%\years{2015} \textsc{Classical Cryptography}, Final work for the information theory Course.\\
%\years{2015} \textsc{Bose-Einstein Condensation in ultra-cold atoms}, Final work for the solid state course.\\
%\years{2016} \textsc{A novel Vacuum State for a massless Scalar Field in globally hyperbolic Spacetimes.} Final work for the quantum field theory course.\\


% \section*{\textcolor{BrickRed}{Academic References}}
 %\textsc{Andres F. Reyes-Lega.} Associate professor physics department, Universidad de los Andes.
 %\href{mailto:anreyes@uniandes.edu.co}{anreyes@uniandes.edu.co}.\\[1ex]
%\textsc{Pedro Bargueño}.Associate professor physics department, Universidad de los Andes. \href{mailto:pbargueno@uniandes.edu.co}{pbargueno@uniandes.edu.co}.\\[1ex]
%\textsc{Carlos Andrés Flórez}. Associate professor physics department, Universidad de los Andes. \href{mailto:ca.florez@uniandes.edu.co}{ca.florez@uniandes.edu.co}.


 \end{document}